\documentclass[10pt]{article}

\usepackage{amsmath, amssymb,amsthm}
\usepackage{multicol}

\usepackage[margin=.5in]{geometry}

\theoremstyle{definition}
\newtheorem*{definition}{Definition}
\theoremstyle{theorem}
\newtheorem*{theorem}{Theorem}

\newcommand{\ds}{\displaystyle}

\pagestyle{empty} %% Suppress page numbers
\setlength\parindent{0pt} %% Supress indentation

\begin{document}
\section*{Sample Statistics}
The sample \(\left\{x_1, x_2, \ldots, x_n\right\}\) is data obtained by taking measurements of some variable from a sample of size \(n\) from the total population \(X\).

\begin{definition}
	The \emph{sample mean} is \(\bar{x}= \dfrac{x_1+x_2+\cdots+x_n}{n}=\dfrac{\sum_{i=1}^{n}x_i}{n}\)
\end{definition}

\begin{definition}
	The \emph{sample median} is the middle value in the ordered sample. If the sample size is even, the average of the middle to values is taken.
\end{definition}

\begin{definition}
	The concept of median can be generalized to the \emph{\(p\)-th percentile}. Where \(p\in (0,1)\) and \(\tilde{x}_p\) will either be the \(p(n+1)\)-th value in the ordered sample, or the weighted contribution of the nearest values, relative to how close they are the integer part of \(p(n+1)\)
\end{definition}

Note: For easier computation of quantiles, just use medians of the lower and upper halves of the ordered sample.

\begin{definition}
	The \emph{sample variance} is \(s^2=\dfrac{1}{n-1}\ds\sum_{i=1}^{n}(x_i - \bar{x})^2 = \dfrac{1}{n-1}\left[\sum_{i=1}^{n}x_i^2 - n\bar{x}^2\right]\). And the \emph{sample standard deviation (SD)} is \(s=\sqrt{s^2}\).
\end{definition}

Note: We like standard deviation because its units match the units of the sample.

\begin{definition}
	The \emph{standardized \(z\)-score} for a data value is \(z_i = \dfrac{x_i - \bar{x}}{s}\)
\end{definition}

\begin{definition}
	If \(x\) and \(y\) are samples of size \(n\), the \emph{sample correlation coefficient} is \(r=\dfrac{s_{xy}}{s_x s_y}\) where
	\begin{align*}
		s_{xy} = \sum_{i=1}^{n}(x_i - \bar{x})(y_i - \bar{y}) = \sum_{i=1}^{n}(x_i y_i - n\bar{x}\bar{y})
	\end{align*}
\end{definition}
Note: We can also express \(r\) as the average of the products of the standardized \(z\)-scores:
\begin{align*}
	r = \frac{1}{n-1} \sum_{i=1}^{n} \left( \frac{x_i-\bar{x}}{s_x} \right) \left( \frac{y_i-\bar{y}}{s_y} \right)
\end{align*}
When performing linear regression using the \emph{least squares method} we get the line
\begin{align*}
	\frac{y-\bar{y}}{s_y} = r \left[\frac{x-\bar{x}}{s_x}\right] && \text{or} && y = r\frac{s_y}{s_x}x + (\bar{y}-b\bar{x})
\end{align*}

\section*{Hypothesis Testing}
\subsection*{Confidence Intervals}
A confidence interval is an interval about a sample statistic within which we have some confidence that the true population parameter lies.
\begin{align*}
	\text{Point Estimate}&\pm\text{Margin of Error} \\
	\text{Point Estimate}&\pm\text{Critical Value}\cdot\text{Standard Error} \\
	\bar{x} &\pm z_{\alpha/2}\frac{\sigma}{\sqrt{n}} \\
	\hat{p} &\pm z_{\alpha/2}\sqrt{\frac{\hat{p}(1-\hat{p})}{n}}
\end{align*}
\end{document}